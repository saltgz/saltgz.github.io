\documentclass[10pt]{beamer}

\usetheme{metropolis}
\usepackage{appendixnumberbeamer}

\usepackage{booktabs}
\usepackage[scale=2]{ccicons}

\usepackage{pgfplots}
\usepgfplotslibrary{dateplot}

\usepackage{xspace}
\newcommand{\themename}{\textbf{\textsc{metropolis}}\xspace}

\usepackage{amsmath}
\usepackage{amsfonts}
\usepackage{amssymb}
\usepackage{multimedia}
\usepackage[official]{eurosym}
\usepackage{hyperref}
\usepackage[italian]{babel}
\usepackage[T1]{fontenc}
\usepackage[utf8]{inputenc}

\title{Base Knowledge of Statistic}
\subtitle{Artificial Intelligence Master Course}
\date{\today}
\author{PhD Student Stefano Pio Zingaro}
\institute{Department of Computer Science and Engineering \\ Universit\`a degli studi di Bologna}

\begin{document}

	\maketitle

	\begin{frame}{Table of contents}
		\setbeamertemplate{section in toc}[sections numbered]
		\tableofcontents[hideallsubsections]
	\end{frame}

	\metroset{block=fill}

	\begin{frame}{Premessa}

		\begin{columns}
			
			\begin{column}{0.5\textwidth}

			   \begin{block}{Di che cosa parleremo}
       				
       				\begin{itemize}
       					
       					\item Deep learning
       					\item Feature Extraction
       					\item PCA
								\item Supervised Learning
       					\item Unsupervised Learning
       				
       				\end{itemize}
      			
      			\end{block}

			\end{column}

			\begin{column}{0.5\textwidth}
		    	
		    	\begin{block}{Di che cosa NON parleremo}

		        	\begin{itemize}
       					
       					\item Augmented reality
								\item \alert{Big data}
								\item Embodied agent
								\item Logic programming
								\item Multi-agent system
								\item Spatial-temporal reasoning
       				
       				\end{itemize}
		      	
		      	\end{block}

			\end{column}
		
		\end{columns}
	
	\end{frame}

	\section{Formulazione dell'Ipotesi}

	\begin{frame}{Ipotesi}

		\begin{itemize}

			\item Quando si effettua un esperimento vengono formulate delle ipotesi.
			\item Per ognuna delle ipotesi, viene fissata una confidenza.
			\item È possibile valutare quanto i risultati dell'esperimento sono in accordo con l'ipotesi grazie ai test di significatività, calcolando il $p_{value} $(\textit{livello di significatività osservato}).

		\end{itemize}

		\begin{exampleblock}{Esempio}

			Se la confidenza desiderata è 95\%  allora $\alpha = 1 - 0.95 = 0.05$, $\alpha$ viene detto \textit{livello di significatività atteso}

		\end{exampleblock}

	\end{frame}

	\begin{frame}{Workflow}

		\begin{enumerate}

			\item Ipotesi $H$ (non dovuto al caso)\\VS\\Ipotesi $H_0$ (dovuto al caso)
			\item Risultati Sperimentali $X=(x_1,x_2,x_3,...,x_n)$
			\item Probabilità dei risultati data l'ipotesi $P(X|H_0)$
			\item Test di Significatività
			
			\begin{itemize}

				\item $(p_{value}>\alpha)\rightarrow$ \textbf{Non Rigetto} $H_0$
				\item $(p_{value}<=\alpha)\rightarrow$ \textbf{Rigetto} $H_0$

			\end{itemize}

		\end{enumerate}

		\begin{alertblock}{Attenzione}
	
			Se l'ipotesi $H_0$ viene respinta al livello di significatività $5$\%, allora abbiamo il $5$\% di probabilità di respingere un'ipotesi che era vera.
	
		\end{alertblock}
	
		\tiny{Fonte: \url{http://www.quadernodiepidemiologia.it/epi/assoc/t_stu.htm}}

	\end{frame}

	\section{Raccolta dei Dati}

		\begin{frame}{Raccolta dei Dati}

			\begin{exampleblock}{Esempio}
				\textbf{Ipotesi $H$}\\
				“L'età di una persona influisce sulla sua altezza”\\
				\textbf{Domanda}\\
				Quali dati sarà più utile raccogliere?\\
				\begin{enumerate}
					\item Altezza
					\item Età
					\item Sesso
				\end{enumerate}
			\end{exampleblock}
	
		\end{frame}

	\section{Analisi Statistica dei Dati}

	\begin{frame}{Perché l'Analisi Statistica dei Dati?}

		\begin{block}{Esperimento} 
		  Gli esperimenti producono un flusso di DATI. 
		\end{block}
		
		\begin{block}{Interpretazione} 
		  I dati, interpretati, possono dare un significato all'esperimento. Spesso vengono ricercate: 
		  \begin{itemize} 
		  	\item Differenze tra le misurazioni; 
		  	\item Invarianti del sistema studiato.
		  \end{itemize}
		\end{block}
		
		\begin{block}{Analisi} 
		  Diventa necessaria un'attenta Analisi Statistica.
		\end{block}

	\end{frame}

	\begin{frame}{Rappresentazione dei Dati: Gli Istogrammi}

		\begin{figure}[1]
			\includegraphics[scale=0.45]{img/1.png}
			\caption{\small{L'istogramma è una rappresentazione grafica di una distribuzione di frequenza di una certa grandezza, ossia di quante volte in un insieme di dati si ripete lo stesso valore.}}
		\end{figure}

	\end{frame}

	\begin{frame}{Descrizione dei Dati: le Distribuzioni}

		\begin{itemize}
			\item \textbf{Distribuzione Binomiale} Variabili che prendono i loro valori da un insieme discreto (eg. La probabilità di avere $k$ eventi positivi su $n$ eventi indipendenti).
			\item \textbf{Distribuzione Normale} Variabili che prendono i loro valori da un insieme continuo (eg. La probabilità di ottenere valori di $x$ in un intervallo infinitesimo).
		\end{itemize}

	\end{frame}

	\begin{frame}{Distribuzione Binomiale per le Variabili Discrete}

		\begin{alertblock}{Definizione}
	      $P(k)=P(X_1+X_2+...+X_n=k)=\binom{n}{k}p^k(1-p)^{n-k}$
		\end{alertblock}

		\begin{figure}[2]
			\includegraphics[scale=0.4]{img/2.png}
			\caption{\tiny{Distribuzioni binomiali con diversi parametri $p$ e $n$.}}
		\end{figure}

	\end{frame}

	\begin{frame}{Distribuzione Normale per le Variabili Continue}

		\begin{alertblock}{Definizione}
			$f(x)={\frac  {1}{\sigma {\sqrt  {2\pi }}}}\;e^{{-{\frac{\left(x-\mu \right)^{2}}{2\sigma ^{2}}}}}~{\mbox{ con }}~x\in {\mathbb  {R}}.$
		\end{alertblock}

		\begin{figure}[3]
			\includegraphics[scale=0.4]{img/3.png}
			\caption{\tiny{Gaussiana con vari parametri di media $\mu$ e varianza $\sigma^2$}}
		\end{figure}

	\end{frame}

	\subsection{La Significatività}

	\begin{frame}{Test di Significatività Parametrici}

		\begin{alertblock}{Definizione}
			Il test di significatività parametrico prevede il confronto tra due distribuzioni \alert{\textbf{normali}} (eg. medie, varianze). 
		\end{alertblock}

		\begin{itemize}
			\item Confronto delle Medie
			\begin{enumerate}
				\item Z-test \footnote{\tiny{\url{https://it.wikipedia.org/wiki/Test_Z}}}
				\item T-Test \footnote{\tiny{\url{https://it.wikipedia.org/wiki/Test_t}}}
			\end{enumerate}
			\item Confronto delle Varianze
			\begin{enumerate}
				\item ANOVA \footnote{\tiny{\url{https://it.wikipedia.org/wiki/Analisi_della_varianza}}}
				\item Test Chi-Quadrato \footnote{\tiny{\url{https://it.wikipedia.org/wiki/Test_chi_quadrato}}}
			\end{enumerate}
		\end{itemize}

	\end{frame}

	\begin{frame}{Test di Significatività NON Parametrici}

		\begin{alertblock}{Definizione}
			Il test di significatività NON parametrico prevede il confronto tra due distribuzioni \alert{\textbf{NON normali}} (eg. medie, varianze). 
		\end{alertblock}

		\begin{itemize}
			\item Test del Segno \footnote{\tiny{\url{https://it.wikipedia.org/wiki/Test_dei_segni}}}
			\item Test di Wilcoxon \footnote{\tiny{\url{https://it.wikipedia.org/wiki/Test_dei_ranghi_con_segno_di_Wilcoxon}}}
		\end{itemize}

	\end{frame}

	\begin{frame}{Estrazione delle Proprietà Rilevanti}

		\begin{itemize}
			\item Correlazione di Pearson~\footnote{\tiny{\url{https://it.wikipedia.org/wiki/Indice_di_correlazione_di_Pearson}}}
			\item Correlazione di Spearman~\footnote{\tiny{\url{https://it.wikipedia.org/wiki/Coefficiente_di_correlazione_per_ranghi_di_Spearman}}}
			\item Principal Component Analysis (PCA)~\footnote{\tiny{\url{https://it.wikipedia.org/wiki/Analisi_delle_componenti_principali}}}
		\end{itemize}
	
	\end{frame}

	\begin{frame}{PCA: Principal Component Analysis}

		\begin{alertblock}{Definizione}
			L'analisi delle componenti principali è una tecnica per la semplificazione dei dati. Permette di scegliere in quante componenti \textbf{ridurre le dimensioni del problema}.
		\end{alertblock}

		\begin{columns}
			\begin{column}{0.5\textwidth}

				\begin{figure}[4]
					\includegraphics[scale=0.15]{img/figure_3.png}
					\caption{\tiny{Rappresentazione grafica di due classi di punti a $3$ dimensioni}}
				\end{figure}

			\end{column}

			$\rightarrow$

			\begin{column}{0.5\textwidth}

				\begin{figure}[5]
					\includegraphics[scale=0.2]{img/figure_4.png}
					\caption{\tiny{Rappresentazione grafica delle classi precedenti dopo l'applicazione di una PCA a 2 componenti.}}
				\end{figure}

			\end{column}
		\end{columns}

	\end{frame}

	\section{Generalizzazione delle Ipotesi}
		
	\begin{frame}{Metodi di Generalizzazione delle Ipotesi}

		\begin{itemize}
			
			\item Supervised (apprendimento da esempi reali)
				
				\begin{itemize}

					\item Neural Networks \footnote{\tiny{\url{https://it.wikipedia.org/wiki/Rete_neurale_artificiale}}}

					\item Hidden Markov Models \footnote{\tiny{\url{https://it.wikipedia.org/wiki/Modello_di_Markov_nascosto}}}

					\item Support Vector Machines \footnote{\tiny{\url{https://it.wikipedia.org/wiki/Macchine_a_vettori_di_supporto}}}

				\end{itemize}

			\item Unsupervised (o semi-supervised/apprendimento ex-novo)

				\begin{itemize}

					\item Hierarchical clustering \footnote{\tiny{\url{https://it.wikipedia.org/wiki/Clustering_gerarchico}}}
					
					\item K-means \footnote{\tiny{\url{https://it.wikipedia.org/wiki/K-means}}}

				\end{itemize}

		\end{itemize}
	
	\end{frame}

	\section{Python's Machine Learning's Frameworks}

	\begin{frame}{Tensor Flow}

		\begin{itemize}
			\item Libreria disponibile in diversi linguaggi di programmazione per il calcolo scientifico.
			\item Abilita la potenza della GPU (NVIDIA CUDA®) per il calcolo computazionale.
			\item Installazione in Python 2.7 con \textit{pip}:
				\begin{enumerate}
					\item \texttt{pip install tensorflow }
					\item \texttt{pip install tensorflow-gpu}
				\end{enumerate}
			\item Documentazione esaustiva e Tutorials disponibili su \url{https://www.tensorflow.org}.
		\end{itemize}
			
	\end{frame}

	\begin{frame}{Keras}

		\begin{itemize}
			\item Installazione in Python 2.7 con \textit{pip}:
				\begin{enumerate}
					\item \texttt{pip install keras}
				\end{enumerate}
			\item Codice della libreria con esempi su \url{https://github.com/fchollet/keras/tree/master/keras}.
		\end{itemize}
			
	\end{frame}

	\begin{frame}{PyTorch}

		\begin{itemize}
			\item Installazione in Python 2.7 con \textit{pip}:
				\begin{enumerate}
					\item \texttt{pip install \url{http://download.pytorch.org/whl/torch-0.1.11.post5-cp27-none-macosx_10_7_x86_64.whl}}
					\item \texttt{pip install torchvision}
				\end{enumerate}
			\item Documentazione ed esempi di codice su \url{http://pytorch.org/docs/}.
		\end{itemize}
			
	\end{frame}

	\section{Conclusioni}

	\begin{frame}{Summary}

	  Get the source of this work and the demo presentation from

	  \begin{center}\url{stefanopiozingaro.github.io/teaching.html}\end{center}

	  This work \emph{itself} is licensed under a
	  \href{http://creativecommons.org/licenses/by-sa/4.0/}{Creative Commons
	  Attribution-ShareAlike 4.0 International License}.

	  \begin{center}\ccbysa\end{center}

	\end{frame}

	\begin{frame}[standout]
	  Questions?
	\end{frame}

	\appendix

\end{document}